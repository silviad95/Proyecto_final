\documentclass[12 pt]{article}
\usepackage[utf8]{inputenc}
\usepackage[spanish]{babel}
\title{Situación actual de la Hepatitis E}
\author{Silvia Díaz Arco}
\date{14 de Diciembre de 2020}
\begin{document}
\maketitle
\tableofcontents
\part{Resumen}
	\begin{abstract}
		La hepatitis viral es la forma más común de hepatitis. Entre los diferentes tipos posibles, la hepatitis E está emergiendo globalmente, siendo responsable de la mayoría de los casos de hepatitis aguda. El virus de la hepatitis E (VHE) es un virus de genoma ARN de cadena sencilla y polaridad positiva que pertenece al género Orthohepevirus de la familia Hepeviridae. Existen diversos genotipos, de los cuales el 1 y el 2 son patógenos humanos estrictos que se transmiten por la ruta fecal-oral debido a aguas contaminadas. Por ese motivo, son predominantes en los países en vías de desarrollo, con pobres condiciones de saneamiento. Por otro lado, los genotipos 3 y 4 son de transmisión zoonótica, presentando una mayor variedad de huéspedes además del humano, siendo el reservorio principal el cerdo. Estos genotipos han adquirido interés al ser comunes en zonas industrializadas. La infección por VHE suele ser asintomática y autolimitante, pero precisamente por eso, en el mundo desarrollado se han dado casos de transmisión parenteral por donantes sanguíneos. Asimismo, se han documentado casos de hepatitis crónica en pacientes inmunocomprometidos infectados con el genotipo 3 del virus. Para el diagnóstico de la infección, además de tener en cuenta el historial clínico del paciente, se recomienda la combinación de pruebas moleculares y serológicas. El tratamiento se basa en un antiviral no específico, ribavirina, no existiendo apenas terapias alternativas. Para la prevención del virus, solo existe una vacuna licenciada en China (Hecolin®), de la que se ha probado su efectividad, pero no su eficacia hacia el genotipo 3 del virus ni su seguridad ante personas de riesgo (embarazadas, inmunocomprometidos, etc). 
		Palabras clave: Hepatitis, viral, VHE, infección, genotipo	
	\end{abstract}
		
\end{document}